\documentclass[a4paper,11pt]{article}
\usepackage[utf8]{inputenc}
\usepackage[slovene]{babel}
\usepackage[margin=1.3in]{geometry}
\usepackage{fullpage}
\usepackage{setspace}

%opening
\title{\huge{Tok svetlobe v metamaterialih na osnovi \\nematskih tekočin}}
\author{Avtorica: Anja Bregar \and Mentor: Miha Ravnik}

\begin{document}

\onehalfspacing

\maketitle


% METAMATERIALI

Optični metamateriali so zanimivo področje fizike materialov, za katere so značilne izjemne optične lastnosti \cite{wegener-2011-nature-review}.
Osnovna lastnost metamateiralov je njihova umetno ustvarjena zgradba: zgrajeni so iz periodičnih enotskih celic s peroidičnostjo, ki je v splošnem nekajkrat manjša od valovne dolžine svetlobe, ki jo želimo nadzorovati.
Tako se svetloba širi skozi material kot bi bil ta homogen. 
Ker pa lahko specifične geometrijske lastnosti in kemijsko sestavo periodičnih enot prilagajamo, je lahko tudi optični odziv takega kompozita do neke mere nadzorovan.
Od poznih 90tih let se je polje metamaterialov razširilo na mnogo raziskovalnih področij. 
V začetnih letih je bilo veliko zanimanja predvsem za materiale, ki bi za izbrane valovne dolžine imeli lomni količnik negativen. 
Negativni lomni količnik je bil pokazan eksperimentalno za svetlobo od infrardeče do optičnih valovnih dolžin \cite{schultz-2000-first-mtm,zhang-2008-fishnet}. 
Z metamateriali z negativnim lomnim količnikom je bilo zasnovanih in ustvarjenih tudi nekaj privlačnih aplikacij, npr. leče z resolucijo pod uklonsko limito ali t.i. plašči nevidnosti \cite{capasso-2016-metalens,zhang-2015-skin-cloak}. 
A zaradi visokih izgub v kovinskih delcih pri optičnih frekvencah in zaradi težav pri sestavljanju metamaterialov, se je pobuda v zadnjem času preselila od razvoja tridimenzionalnih kovinskih metamaterialov k dielektričnim metamaterialov, ki imajo manjše izgube \cite{jacob-2016-all-dielectric}, in metapovršinam --dvodimenzionalnim metamaterialom \cite{capasso-2014-flat-optics-metasurface}. 
Z metapovršinami je mogoče natančno načrtovati fazo in polarizacijo prepuščene svetlobe \cite{capasso-2014-flat-optics-metasurface} in jih uporabljati kot selektivne absorberje.
Težko je zasnovati optično popolnoma izotropen metamaterial, a optična anizotropija je bila tudi izkoriščena na področju hiperboličnih metamaterialov \cite{kivshar-2013-hyperbolic}, katerih dielektrični tenzor ima eno od lastnih vrednosti negativno. 
Z njimi je mogoče doseči negativni lom Poyntingovega vektorja, ne da bi bilo zadoščeno zahtevi po negativni magnetni permeabilnosti. 


% TEKOČI KRISTALI

Še ena velika družina optičnih materialov so tekoči kristali, ki jih danes uporabljamo v različnih optičnih in fotoničnih aplikacijah, posebej v tekočekristalnih zaslonih. 
So mehki materiali, sestavljeni iz paličastih molekul, ki so v določenem območju temperatur in molekulskih koncentracij do neke stopnje orientacijsko urejene, a pozicijsko neurejene \cite{degennes}.
Iz njihove orientacijske urejenosti izvirajo številne njihove značilnosti. 
Ena od njih je njihova anizotropna elastičnost: prosta energija tekočih kristalov se dvigne, če z zunanjimi polji ali robnimi pogoji zmotimo njihov orientacijski red. 
Strukturne in prostorsko odvisne sile, ki sledijo iz elastične proste energije, ustvarjajo množico struktur in pojavov urejanja znotraj direktorskega polja.
Nadalje iz orientacijskega reda sledi tudi optična anizotropija \cite{hecht-optics}: večinoma je izredni lomni količnik (vzdolž dolge osi molekul tekočega kristala) večji od rednega. 
Optična dvolomnost tekočih kristalov je uporabno orodje za nadzor toka svetlobe v optičnih sistemih. 
Ker je tekoče kristale relativno lahko manipulirati z znanjimi polji, še posebej električnim poljem, ki obrne dolgo os tekočekristalnih molekul vzdolž polarizacije električnega polja, je možno nadzirati elastične in optične lastnosti. 
Dandanes se tekoče kristale raziskuje v različnih smereh, kot npr. usmerjeno urejanje koloidnih kristalov \cite{zumer-2006-assembly,musevic-2013-assembly}, zapletena topološka stanja \cite{musevic-2011-knots,smalyukh-2014-knots,zumer-2014-knots} in mikrofluidika \cite{yeomans-2013-lc-microfluidics}.
V širšem kontekstu fotonike so aplikacije raznolike na področjih senzorike \cite{abbott-2013-sensing}, nastavljivih leč \cite{neyts-2017-lensing,lin-2011-lensing}, nastavljivih valovodov \cite{kitzerow-2014-fibres}, in nastavljivih laserjev \cite{humar-2016-lasing}.


% LC WITH MTM 

Področji tekočih kristalov in metamaterialov sta prepleteni s številnimi eksperimentalnimi in teoretičnimi primeri. 
Vredno je omeniti, da negativni lom ni omejen na področje metamaterialov: tekoči kristali in drugi dvolomni materiali lahko lomijo svetlobo negativno, ko svetloba na vzorec vpada skoraj pravokotno \cite{lavrentovich-2006-lc-neg}. 
Holesterične tekočekristalne tanke plasti je mogoče uporabiti kot metapovršine zaradi njihove zmožnosti učinkovitega nadzora nad fazo prepuščene svetlobe \cite{ozaki-2016-patterned-lc}. 
S tekočimi kristali je bila dosežena orientacijska in do neke mere celo pozicijska urejenost sestavnih gradnikov metamateriala: izvedeni so bili številni eksperimenti z zlatimi nanopaličicami, ki so se zaradi svoje podolgovate oblike in elastičnosti tekočega kristala uredile \cite{lavrentovich-2008-gold-nanorods,smalyukh-2010-self-alignment,lavrentovich-2009-nanorods}. 
Z numeričnimi analizami je bilo pokazano, da bi bile lastne vrednosti tenzorja dielektričnosti materiala, sestavljenega iz kovinskih kroglic, razpršenih v tekočem kristalu, predznačeni nasprotno \cite{xuan-2013-nanoparticle-lc,khoo-2014-nanoparticle-lc}. 
V tej smeri so bili opravljeni tudi ekspermienti, ki kažejo, da imajo prevlečene zlate nanosfere, razpršene v nematskem tekočem kristalu, negativno redno in pozitivno izredno lastno vrednost tenzorja dielektričnosti pri nizkih frekvencah \cite{goodby-2011-lc-gold-mtm}.
Druga možnost uporabe tekočih kristalov znotraj metamaterialov je nastavljanje lastnosti metamateriala z električnim poljem. 
Ko tekočekristalni metamaterial priključimo na napetost, lahko natančno uravnavamo frekvenco, pri kateri imamo željene metamaterialne lastnosti \cite{zhang-2007-lc-mtm-tuning,shalaev-2007-tunable-lc,baets-2011-ring-resonators}, tekoče kristale pa je mogoče uporabiti tudi za vklapljanje in izklapljanje metapovršin \cite{buchnev-2015-lc-mtm-switch}. 

Osnovna motivacija te disertacije je preučevati vpliv optične anizotropije na odziv metamateriala na svetlobo. 
Anizotropija bi bila lahko značilnost metamateriala kot celote ali pa lastnost nekaterih njegovih sestavnih delov. 
Z preučevanjem anizotropije se približujemo drugemu relevantnemu še odprtemu izzivu: ustvariti mehke metamateriale, katerih lastnosti bi bilo mogoče nastavljati z zunanjimi polji in parametri. 
Kompleksne nematske tekočine, morda v povezavi s koloidnimi vključki, lahko dosežejo takšne anizotropne nadzorske strukture. 
Tako je končni cilj te disertacije raziskati možnosti za implementacjio nastavljivih mehkih metamaterialov.


% REFRACTION INTO HYPERBOLIC METAMATERIALS

Bolj konkretno se bomo ukvarjali z optično anizotropnimi sistemi, katerih lastni vrednosti tenzorja dielektričnosti sta predznačeni različno: z metamateriali, analognimi tekočim kristalom. 
Zanimal nas bo lom iz vakuuma v metamaterialni sistem z optično osjo, ki je homogena, a nagnjena glede na mejo med snovema. 
Kot nadgradnjo tega sistema bomo preučevali primer periodične modulacije direktorskega profila. 
Kot primerjavo z že obsoječim holesteričnim sistemom s pozitivnim lomnim količnikom, kjer se optična os zvija v ravnini, vzporedni z mejo med vakuumom in metamaterialom, bomo raziskovali holesterične hiperbolične metamaterialne sisteme eno ali dvema negativnima lastnima vrednostma dielektričnega tenzorja.
Še posebej nas bo zanimal režim Braggovega odboja za krožno polarizirano svetlobo. 
Simulirali bomo tudi širjenje svetlobe pod kotom glede na optično os, za kar še ni jasnega teoretičnega modela. 
V splošnem bomo modulirali prostorsko spremenljiv in frekvenčno disperzen dielektrični tenzor in raziskali možnosti za učinkovito upravljanje s propagacijo svetlobe. 


% WAVEGUIDING 

Nadzor nad tokom svetlobe bomo v kontekstu aplikacij kot so valovni vodniki in leče raziskovali tudi za materiale z pozitivnim lomnim količnikom. 
Zanimiva priložnost za uporabo v valovnih vodnikih je, da se njihove lastnosti ojača s tekočimi kristali. 
V sodelovanju s prof. Etiennom Brasseletom z Univerze v Bordeauxu bomo simulirali in teoretično analizirali različne predloge za vaovne vodnike, napolnjene s tekočimi kristali. 
Z različnimi robnimi pogoji in drugimi tehnikami je mogoče v tekočekristalnih valovnih vodnikih vzpostaviti različne direktorske profile. 
Prostorsko spreminjanje lomnega količnika lahko nato vpliva na propagacijo svetlobe. 
Z določenimi direktorskimi profili lahko dosežemo fokusiranje izbranih vpadnih polarizacij svetlobe. 
Poseben poudarek bomo dali direktorskim profilom, ki svetlobo fokusirajo z pozitivno anizotropnimi tekočimi kristali ($n_o \leq n_e$), tako da so strukture eksperimentalno lažje dosegljive. 


% ORDERING OF COLLOIDS

S ciljem ustvarjanja samourejajočih se gradnikov v tridimenzionalnih metamaterialih, bi si radi ogledali urejanje kovinskih koloidnih delcev v tekočem kristalu. 
Ker so nekatere orientacije koloidnih delcev zaradi nematske proste energije tekočega kristala energijsko bolj ugodne \cite{musevic-2013-assembly,smalyukh-2009-assembly}, je mogoče, da se koloidni delci podkvaste oblike samouredijo v dvo- in trodimenzionalne fotonske kristale. 
Razmerje med vplivi geometrije takšnega koloidnega metamateriala in optičnimi parametri snovi, ki jih uporabimo za koloide, tvori bogat in kompleksen raziskovalni sistem.
Primerjava kovinskih koloidnih delcev z dielektričnimi pridoda k eksperimentalni raznolikosti. 
Ker so optični parametri kovin močno odvisni od frekvence uporabljene svetlobe, dosegamo z različnimi valovnimi dolžninami kontrasten metamaterialni odziv. 
Eksperimentalna uresničitev teh idej bi potekala v mogočem sodelovanju s skupino prof. Muševiča na IJS-ju. 


% KODA

Propagacija svetlobe skozi metamateriale je zaradi različnih materialnih in geometrijskih lastnosti metamaterialnih gradnikov zelo zapletena. 
Tudi v tekočih kristalih je lahko zaradi zapletenih direktorskih struktur točen izračun toka svetlobe težaven. 
V tej luči so numerične simulacije ključno orodje v prepoznavanju odziva tovrstnih mehkih struktur. 
Simulacije za to disertacijo bomo izvedli s programsko kodo, temelječo na metodi FDTD, ki je implementirana znotraj skupine. 
Pri metodi FDTD (krajše za angl. finite-difference time-domain) \cite{taflove1}, integriramo Maxwellove enačbe eksplicitno, zaradi česar je mogoče različne optične pojave enostavneje obravnavati. 
Diferencialne enačbe integriramo na Yeejevi mreži, v kateri so komponente električnega in magnetnega polja postavljene v kubični zamaknjeni rešetki in prav tako časovno propagirane z medsebojnim zamikom. 
Naša koda je prilagojena za optično anizotropne materiale in tudi za anizotropne in frekvenčno disperzne materiale s prilagojeno metodo ADE (angl. auxiliary differential equation \cite{taflove1}). 
Frekvenčno disperzno funkcijo snovi lahko opišemo s plazemskim, Drudejevim ali Lorentzovim modelom. 

MEdtem ko je glavni cilj te disertacije raziskati optični odziv mehkih metamaterialov, je nujno upoštevati tudi relaksacijo tekočih kristalov zaradi optičnih električnih polj. 
To bo opravljeno z algoritmi za minimizacijo proste energije \cite{ravnik-2009-lc-modelling}, ki je prav tako implementiran znotraj naše skupine. 
Programska koda upošteva različne časovne skale za širjenje svetlobe ($\approx \mathrm{ps}$) in relaksacijo tekočega kristala ($\approx \mathrm{ms}$). 

Če povzamemo, v tej disertaciji bomo opisali konceptualno različne kombinacije optičnih metamaterialov in tekočih kristalov, katerih skupna točka bo optična anizotropija. 
Po eni strani se bomo ukvarjali s homogeniziranimi materiali z izrednimi lastnostmi, po drugi strani pa z načrtovanjem mehkih koloidnih metamaterialov. 
S tem delom bi radi predlagali nov pogled na nadzor in manipulacijo svetlobe. 

\bibliographystyle{amsplain_copar}
\bibliography{dispozicija}

\end{document}
