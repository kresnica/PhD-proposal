\documentclass[a4paper,11pt]{article}
\usepackage[utf8]{inputenc}
\usepackage[slovene]{babel}
\usepackage[margin=1.3in]{geometry}
\usepackage{fullpage}
\usepackage{setspace}

%opening
\title{\huge{Tok svetlobe v metamaterialih na osnovi \\nematskih tekočin}}
\author{Avtorica: Anja Bregar \and Mentor: Miha Ravnik}

\begin{document}

\setcounter{page}{5}

\onehalfspacing

\maketitle


% METAMATERIALI

Optični metamateriali so zanimivo področje fizike materialov, za katere so značilne izjemne optične lastnosti \cite{wegener-2011-nature-review}.
Osnovna lastnost metamaterialov je njihova umetno ustvarjena zgradba: zgrajeni so iz periodičnih enotskih celic, ki so v splošnem nekajkrat manjše od valovne dolžine svetlobe, ki jo želimo nadzorovati.
Tako se svetloba širi skozi material, kot bi bil ta homogen. 
Ker pa lahko specifične geometrijske lastnosti in kemijsko sestavo periodičnih enot prilagajamo, je lahko tudi optični odziv takega kompozita do neke mere nadzorovan.
Od poznih 90tih let se je polje metamaterialov razširilo na mnogo raziskovalnih področij. 
V začetnih letih je bilo veliko zanimanja predvsem za materiale, katerih lomni količnik bi bil za izbrane valovne dolžine negativen. 
Primeri materialov z negativnim lomnim količnikom so bili pokazani eksperimentalno za svetlobo od infrardeče do optičnih valovnih dolžin \cite{schultz-2000-first-mtm,zhang-2008-fishnet}. 
Z njimi je bilo zasnovanih in ustvarjenih tudi nekaj privlačnih aplikacij, npr. leče z resolucijo pod uklonsko limito \cite{capasso-2016-metalens} in t.i. plašči nevidnosti \cite{zhang-2015-skin-cloak}. 
A zaradi visokih izgub znotraj kovinskih delcev pri optičnih frekvencah in zaradi težav pri sestavljanju metamaterialov se je pobuda v zadnjem času preselila od razvoja tridimenzionalnih kovinskih metamaterialov k dielektričnim metamaterialom \cite{jacob-2016-all-dielectric}, katerih izgube so manjše, in metapovršinam -- dvodimenzionalnim metamaterialom \cite{meinzer-2014-metasurface}. 
Z metapovršinami je mogoče natančno načrtovati fazo in polarizacijo prepuščene svetlobe \cite{capasso-2014-flat-optics-metasurface} in jih uporabljati kot selektivne absorberje.
Zasnovani so bili tudi metamateriali s poudarjeno optično anizotropijo -- hiperbolični metamateriali \cite{kivshar-2013-hyperbolic}, katerih dielektrični tenzor ima eno od lastnih vrednosti negativno. 
Z njimi je mogoče doseči negativni lom Poyntingovega vektorja, ne da bi bilo potrebno vplivati na vrednost magnetne permeabilnosti.


% TEKOČI KRISTALI

Naslednja velika skupina optičnih materialov so tekoči kristali, ki jih danes uporabljamo v različnih optičnih in fotonskih aplikacijah, posebej v tekočekristalnih zaslonih. 
So mehki materiali, sestavljeni iz paličastih molekul, ki so v določenem območju temperatur in molekulskih koncentracij orientacijsko urejene, a pozicijsko neurejene \cite{degennes}.
Iz njihovega orientacijskega reda izvirajo številne njihove lastnosti. 
Ena od njih je njihova anizotropna elastičnost: če z zunanjimi polji ali robnimi pogoji zmotimo orientacijski red, se prosta energija tekočih kristalov zviša. 
Strukturne in prostorsko odvisne sile, ki sledijo iz elastične proste energije, ustvarjajo množico različnih struktur in pojavov urejanja znotraj direktorskega polja.
Nadalje iz orientacijskega reda sledi tudi optična anizotropija \cite{hecht-optics}: večinoma je izredni lomni količnik (vzdolž dolge osi molekul tekočega kristala) večji od rednega. 
Optična dvolomnost tekočih kristalov je uporabno orodje za nadzor toka svetlobe v optičnih sistemih. 
Ker je na tekoče kristale relativno lahko vplivati z zunanjimi polji, še posebej električnim poljem, ki obrne dolgo os tekočekristalnih molekul vzdolž polarizacije električnega polja, je možno nadzirati elastične in optične lastnosti tekočih kristalov. 
Dandanes se tekoče kristale raziskuje v različnih smereh, kot so npr. usmerjeno urejanje koloidnih kristalov \cite{zumer-2006-assembly,musevic-2013-assembly}, zapletena topološka stanja \cite{musevic-2011-knots,smalyukh-2014-knots,zumer-2014-knots} in mikrofluidika \cite{yeomans-2013-lc-microfluidics}.
V širšem kontekstu fotonike aplikacije segajo na področja npr. senzorike \cite{abbott-2013-sensing}, nastavljivih leč \cite{neyts-2017-lensing,lin-2011-lensing}, nastavljivih valovodov \cite{kitzerow-2014-fibres}, in nastavljivih laserjev \cite{humar-2016-lasing}.


% LC WITH MTM 

Področji tekočih kristalov in metamaterialov sta prepleteni s številnimi eksperimentalnimi in teoretičnimi primeri. 
Vredno je omeniti, da negativni lom ni omejen na področje metamaterialov: tekoči kristali in drugi dvolomni materiali lahko lomijo svetlobo negativno, ko svetloba na vzorec vpada skoraj pravokotno \cite{lavrentovich-2006-lc-neg}. 
Holesterične tekočekristalne tanke plasti je mogoče uporabiti kot metapovršine zaradi njihove zmožnosti učinkovitega nadzora nad fazo prepuščene svetlobe \cite{ozaki-2016-patterned-lc}. 
S tekočimi kristali je bila dosežena orientacijska in do neke mere celo pozicijska urejenost sestavnih gradnikov metamateriala: izvedeni so bili eksperimenti z zlatimi nanopaličicami, ki so se zaradi svoje podolgovate oblike in elastičnosti tekočega kristala uredile \cite{lavrentovich-2008-gold-nanorods,smalyukh-2010-self-alignment,lavrentovich-2009-nanorods}. 
Numerično pridobljeni lastni vrednosti tenzorja dielektričnosti materiala, sestavljenega iz kovinskih kroglic, razpršenih v tekočem kristalu, sta bili predznačeni nasprotno \cite{xuan-2013-nanoparticle-lc,khoo-2014-nanoparticle-lc}. 
Eksperimentalna dela kažejo, da imajo prevlečene zlate nanosfere, razpršene v nematskem tekočem kristalu, negativno redno in pozitivno izredno lastno vrednost tenzorja dielektričnosti pri nizkih frekvencah \cite{goodby-2011-lc-gold-mtm}.
Druga možnost uporabe tekočih kristalov znotraj metamaterialov je nastavljanje lastnosti metamateriala z električnim poljem. 
Ko tekočekristalni metamaterial priključimo na napetost, lahko uravnavamo frekvenco, pri kateri imamo željene metamaterialne lastnosti \cite{zhang-2007-lc-mtm-tuning,shalaev-2007-tunable-lc,baets-2011-ring-resonators}, tekoče kristale pa je mogoče uporabiti tudi za vklapljanje in izklapljanje metapovršin \cite{buchnev-2015-lc-mtm-switch}. 

Osnovna motivacija te disertacije je preučevati vlogo optične anizotropije pri širjenju svetlobe skozi metamaterial. 
Anizotropija je lahko v tem kontekstu značilnost metamateriala kot celote ali pa lastnost nekaterih njegovih sestavnih delov. 
S preučevanjem odziva na optično anizotropije se približujemo drugemu relevantnemu še odprtemu izzivu: ustvariti mehke metamateriale, katerih lastnosti bi bilo mogoče nastavljati z zunanjimi polji in parametri. 
Tovrstne anizotropne strukture, ki bi jih bilo mogoče nadzorovati, lahko dosežemo s kompleksnimi nematskimi tekočinami, morda v povezavi s koloidnimi vključki.
Tako je končni cilj te disertacije raziskati možnosti za implementacijo nastavljivih mehkih metamaterialov.


% REFRACTION INTO HYPERBOLIC METAMATERIALS

Natančneje, ukvarjali se bomo z optično anizotropnimi sistemi, katerih lastni vrednosti tenzorja dielektričnosti sta predznačeni različno: z metamateriali, analognimi tekočim kristalom. 
Zanimal nas bo lom iz vakuuma v metamaterialni sistem z optično osjo, ki je homogena, a nagnjena glede na mejo med snovema. 
Kot nadgradnjo tega sistema bomo preučevali primer periodične modulacije direktorskega profila. 
Že obstoječe holesterične sisteme s pozitivnim lomnim količnikom, pri katerih se optična os zvija v ravnini, vzporedni z mejo med vakuumom in holesteričnim tekočim kristalom, bomo primerjali s holesteričnimi hiperboličnimi metamaterialnimi sistemi z eno ali dvema negativnima lastnima vrednostma dielektričnega tenzorja.
Še posebej nas bo zanimal režim Braggovega odboja za krožno polarizirano svetlobo. 
Simulirali bomo tudi širjenje svetlobe pod kotom glede na spreminjajočo se optično os, za kar še ni jasnega teoretičnega modela. 
V splošnem bomo modulirali prostorsko spremenljiv in frekvenčno disperzen dielektrični tenzor in raziskali možnosti za učinkovito upravljanje s propagacijo svetlobe. 


% WAVEGUIDING 

Nadzor nad tokom svetlobe bomo v kontekstu aplikacij kot so valovni vodniki in leče raziskovali tudi pri materialih s pozitivnim lomnim količnikom. 
Zanimiva priložnost za uporabo v valovnih vodnikih je ojačanje njihovih lastnosti s tekočimi kristali. 
V sodelovanju s prof. Etiennom Brasseletom z Univerze v Bordeauxu bomo simulirali in teoretično analizirali različne predloge za valovne vodnike, napolnjene s tekočimi kristali. 
Z različnimi robnimi pogoji in drugimi tehnikami je mogoče v tekočekristalnih valovnih vodnikih vzpostaviti različne direktorske profile. 
Prostorsko spreminjanje lomnega količnika lahko nato vpliva na propagacijo svetlobe. 
Z določenimi direktorskimi profili lahko dosežemo fokusiranje izbranih vpadnih polarizacij svetlobe. 
Posebej se bomo posvetili direktorskim profilom, ki svetlobo fokusirajo z pozitivno anizotropnimi tekočimi kristali ($n_o \leq n_e$), zaradi česar so strukture eksperimentalno lažje dosegljive. 


% ORDERING OF COLLOIDS

S širšim ciljem ustvariti tridimenzionalni metamaterial iz samourejajočih se osnovnih gradnikov bi si radi ogledali urejanje kovinskih koloidnih delcev v tekočem kristalu. 
Ker so nekatere orientacije koloidnih delcev zaradi nematske proste energije tekočega kristala energijsko bolj ugodne \cite{musevic-2013-assembly,smalyukh-2009-assembly}, je mogoče, da se koloidni delci podkvaste oblike samouredijo v dvo- in trodimenzionalne fotonske kristale. 
Razmerje med vplivom geometrijskih lastnosti takšnega koloidnega metamateriala in vplivom optičnih parametrov snovi, iz katerih so koloidi, tvori bogat in kompleksen raziskovalni sistem.
Ker so optični parametri kovin močno odvisni od frekvence uporabljene svetlobe, dosegamo z različnimi valovnimi dolžninami kontrasten metamaterialni odziv. 
Eksperimentalna uresničitev teh idej bi potekala v mogočem sodelovanju s skupino prof. Muševiča na IJS-ju. 


% KODA

Širjenje svetlobe skozi metamateriale je zaradi različnih snovnih in geometrijskih lastnosti metamaterialnih gradnikov zelo zapletena. 
Tudi v tekočih kristalih je lahko zaradi zapletenih direktorskih struktur točen izračun toka svetlobe težaven. 
V tej luči so numerične simulacije ključno orodje za prepoznavanje odziva metamaterialnih nematskih struktur. 
Simulacije za to disertacijo bomo izvedli s programsko kodo, ki temelji na metodi FDTD in je implementirana znotraj skupine. 
Pri metodi FDTD (krajše za angl. finite-difference time-domain) \cite{taflove1}, integriramo Maxwellove enačbe eksplicitno, zaradi česar je mogoče različne optične pojave obravnavati enostavneje. 
Diferencialne enačbe integriramo na Yeejevi mreži, v kateri so komponente električnega in magnetnega polja postavljene v kubični zamaknjeni rešetki in časovno propagirane z medsebojnim časovnim zamikom. 
Naša koda je prilagojena za optično anizotropne materiale in tudi za anizotropne in frekvenčno disperzne materiale s prilagojeno metodo ADE (angl. auxiliary differential equation \cite{taflove1}). 
Frekvenčno disperzno funkcijo snovi lahko opišemo s plazemskim, Drudejevim ali Lorentzovim modelom. 

Medtem ko je glavni cilj te disertacije raziskati optični odziv mehkih metamaterialov, je nujno upoštevati tudi relaksacijo tekočih kristalov zaradi optičnih električnih polj. 
To bo opravljeno z algoritmi za minimizacijo proste energije \cite{ravnik-2009-lc-modelling}, ki so prav tako implementirani znotraj naše skupine. 
Programska koda upošteva različne časovne skale za širjenje svetlobe ($\approx \mathrm{ps}$) in relaksacijo tekočega kristala ($\approx \mathrm{ms}$). 

Če povzamemo, v tej disertaciji bomo opisali konceptualno različne kombinacije optičnih metamaterialov in tekočih kristalov, katerih skupna točka bo optična anizotropija. 
Po eni strani se bomo ukvarjali s homogeniziranimi materiali z izrednimi lastnostmi, po drugi strani pa z načrtovanjem mehkih koloidnih metamaterialov. 
S tem delom bi radi predlagali nov pogled na nadzor in manipulacijo svetlobe. 

\bibliographystyle{amsplain_copar_slo}
\bibliography{dispozicija}

\end{document}
