\documentclass[a4paper,11pt]{article}
\usepackage[utf8]{inputenc}
\usepackage[margin=1.3in]{geometry}
\usepackage{fullpage}
\usepackage{setspace}

%opening
\title{\huge{Soft matter metamaterials}}
\author{Author: Anja Bregar \and Advisor: Miha Ravnik}

\begin{document}

\onehalfspacing

\maketitle

% \begin{abstract}
% In my PhD thesis, I will highlight different phenomena connected with light propagation in soft matter-like meta-materials. 
% \end{abstract}
% 
% 
% 
% ------------------- State of the art -----------------
%


% LIQUID CRYSTALS

Liquid crystals are soft materials, which in a range of temperatures and molecular concentrations posess a degree of orientational order in contrast to their positional disorder \cite{degennes}. 
This is typically achieved by the molecules of liquid crystal being elongated in one direction.  
They have proven to be an important material in soft matter physics with many possibilities for applications in different industries as well, notably in liquid crystal displays.
Their orientational order is the origin of several other characteristics. 
One of them is their anisotropic elasticity: the free energy of liquid crystals increases, if -- because of external fields or boundary conditions -- their orientational order is disturbed. 
The structural and spatially dependent forces, which arise from that account, generate a plethora of field structures and ordering phenomena. 
%With the elasticity, they can take into account the orientational order of colloidal inclusions. 
Furthermore, their orientational order implies optical anisotropy \cite{hecht-optics}: mostly, liquid crystals possess an extraordinary index of refraction (along the long axis of the molecules) larger than the ordinary index. 
Their optical birefringence is an noteworthy tool in the control of flow-of-light in optical systems. 
Since liquid crystals are also relatively easily manipulated with external fields, especially electric field, which tends to turn the long axis of the liquid crystal molecules to be polarized along the field, the elastic and optical properties can be well controlled.
%With the electric field, the elastic forces and the optical properties can be controlled. 




% METAMATERIALS


Metamaterials are another fascinating segment of materials science. 
They are characterised by their extraordinary optical properties \cite{gja}.
(We would have to add that the concept of metamaterials applies to any kind of waves; besides optical waves, also elastic or acoustic waves come into play). 
The defining feature of metamaterials is their man-made composition: they are built in periodic unit cells with their periodicity being generally much smaller than the wavelength of light. 
Since the building blocks are smaller than the wavelength, the light propagates through the material as if it was homogeneous (anisotropic, but homogeneous). 
However, since the geometry and the material composition of the building blocks can be adjusted, the optical response of such a composite can be, to an extent, engineered. 

The field of research has opened around year 2000 \cite{koak} and has so far expanded to different areas of investigation. 
The central interest at the beginning was to obtain materials which could produce negative refractive index for light of a given wavelength. 
It has been done for many different wavelengths \cite{pwqo}. 
Another dream was to fabricate an invisibility cloak or a lens with subwavelength resolution. 
Some of the work in that direction has been successfully done \cite{lal, ei}. 
Because of the high losses inside metals at optical frequencies, the incentive has moved from producing bulk metamaterials towards other options, namely dielectric metamaterials, which have lower losses \cite{}, and metasurfaces, which are 2D metamaterials. 
An use of the metamaterials is as selective absorbers for selected incoming angles, polarisations and wavelengths of light. 
Lately, the research focus has shifted towards wavefront shaping -- the shaping of the phase and the polarisation of the transmitted wave -- with metasurfaces \cite{}.  
Another direction in which the field has departed are hyperbolic metamaterials: materials with anisotropic permittivity tensor, which may also have some of its eigenvalues negative. 
With it, negative refraction can also be achieved in the sense of the Poynting vector being refracted negatively.


% LC WITH MTM 

Metamaterials are intrinsically connected with the field of liquid crystals, as both are optically anisotropic. 
In addition, the optical anisotropy could be in liquid crystal metamaterials enforced even further and more pronounced. 
Liquid crystals (and other birefringent materials) can by itself, without metamaterial inclusions, for a narrow region of incoming angles around perpendicular to the surface boundary, exhibit negative refraction, since the walk-off angle is negative for birefringent media with positive anisotropy. 
%It is not a true negative-index metamaterial. 
%Anisotropic metamaterials with hyperbolic dispersion relation can exhibit all possible angles of the Poynting vector. (to boš ti delala, ne o tem tu pisat.)
Another challenge in the field of metamaterials is the production of bulk metamaterials. 
In the past, the engineering of metamaterials has mostly taken the up-to-the-bottom approach, meaning that every building block has been specifically placed in its place, which is experimentally very difficult and renders many applications impractical. 
It would be very much desired to take a bottoms-up approach of self-arranging colloidal particles as in \cite{tartan}. 
There, the colloidal particles, obtained from direct laser writing, were self-assembled in liquid crystal. 
Some suggestions to obtain hyperbolic metamaterials have also shown the way of metallic (plasmonic) spheres, immersed in liquid crystal, which partially order due to the liquid crystal elasticity. 
Numerical analysis (homogenisation techniques) show that the eigenvalues of such media would be differently signed \cite{khoo}. 
Some experiments have been performed \cite{ponsinet-virginie?}. 
Metamaterial building blocks can also mimic liquid crystals in order to achieve their orientational order. 
Many experiments have been performed with golden nanorods ordering because of their prolong shape in connection with the elasticity of liquid crystals \cite{nanoparticles}. 
Some experiments have been very explicit in this mimicking \cite{shadrivov}. 
As it comes very naturally with liquid crystals, many applications also include the tuning of the director field with electric field. 
With the help of voltage, the resonant frequency, i.e. the operational frequency at which negative refractive index is observed, was fine-tuned, and the liquid crystals were used to switch between the on- and off-states of a metasurface. 

A way to engineer soft (meta-)materials is to combine liquid crystal with metallic colloids. 
The elasticity of liquid crystals affects the distribution and arrangement of colloids. 
The colloidal suspension in liquid crystals have been shown to be ordered because of the elasticity of liquid crystals \cite{1, 2, 3,}: 
Musevic and Smalyukh groups have experimentally shown the elastic forces between the defect structures to be strong enough to achieve orientational ordering of the (spatially anisotropic, i.e. square or triangular) colloids, and also to lower the free energy in such colloidal clusters by arranging. 
Theoretical work has been done by our group as well \cite{cc}. 

A crucial reason why to control and manipulate light is to guide it. 
Many different applications, from photonic circuits to waveguides, all designed to convey information, rely on efficient control of light. 
An compelling option for waveguiding is to enhance their properties with liquid crystals. 
Liquid crystals can be used as an addition to the existing photonic-crystal-like waveguides \cite{dd}, where it has been shown they increase the performance of such structure by XX\%. 
If the photonic fibre is filled exclusively with liquid crystal, it can be used as a waveguide as well, since the refractive-index gradient controls the flow-of-light. 

% KODA

Optical propagation through metamaterials is very complicated due to different material effects and elaborate geometry boundary conditions. 
In liquid crystals, the definite calculation of the flow of light is equally challenging, because of their optical birefringence. 
With that in mind, numerical simulations are a viable tool in determining the response of such soft matter structures. 
My thesis will be prepared with a code developed in our group. 
It is based on FDTD with slight alterations. 
FDTD (short for finite-difference time-domain) is a finite difference method, where the derivatives are approximated with finite differences, while the time-domain part means a wide range of frequencies can be covered with one simulation. (prehodni pojavi). 
In FDTD, Maxwell's equations are integrated quite explicitly, which allows for simple implementation of different optical phaenomena such as nonlinearity. 
Traditionally, FDTD is used in combination with Yee mesh, in which the components of electrical and magnetic fields are positioned in a cubic stacked grid and propagated in a leapfrog fashion in time. 
Our code is adapted for optically anisotropic materials, as well as for frequency dispersive materials with the ADE (auxiliary differential equation) method. 
As a model for frequency dispersive dielectric function, the plasma, Drude, and Lorentz models can be used. 
We have also implemented the possibility of anisotropic and frequency dispersive materials in one. 

While the main objective of this PhD is to explore the optical response of soft matter metamaterials, it is also crucial to take into account the relaxation of the liquid crystal by optical fields as well. 
This will be done with free energy minimisation algorithms, namely the relaxation algorithm \cite{ravnik}, also incorporated in our group. 
The code takes into account different time scales of the light propagation ($\approx \mathrm{ps}$) and relaxation ($\approx \mathrm{ms}$). 

% REFRACTION INTO HYPERBOLIC METAMATERIALS

With these powerful tools we will examine several theoretical and experimental example cases of soft-matter metamaterials. 
As a model case we will firstly explore optically anisotropic systems with the eigenvalues of the permittivity tensor with different signs: liquid-crystal like metamaterials. 
Firstly we will be interested in a case on refraction from vacuum into a system with optical axis homogeneous across the material and tilted with respect to the metamaterial boundary. 
As an upgrade of this systems, examples with periodic modulation of the director profile will be examined. 
As a comparison with the already existing cholesteric positive-$n$ systems (the optical axis is twisting in the plane of the incoming plane wave), we will explore cholesteric hyperbolic metamaterial systems with some of the permittivity eigenvalues negative and compare results with the all-positive case. 
We will be especially interested in the Bragg-reflective regime for circularly polarised light. 
Additionally, we will perform calculations for the propagation of light perpendicularly to the optical axis, for which no clear theoretical framework is established. 
If the time will permit, we will exhibit 3D spatial modulation of frequency-dispersive permittivity tensor and explore possibilities for successful modulation of light.  

% WAVEGUIDING 

The control over the flow of light will be explored for materials with all-positive refractive index as well. 
In cooperation with prof. Etienne Brasselet from University of Bordeaux, different proposals for waveguides, composed and filled with liquid crystals, will be simulated and analysed theoretically. 
Through different boundary conditions and other manipulation, different director profiles can be introduced into the liquid crystal. 
The modulation of the refractive index profile can in turn affect the propagation of light. 
With some director profiles we can achieve focusing of selected incoming polarizations.
Special emphasis will be put on the director profiles which exhibit focusing for positive anisotropy ($n_{o} < n_e$), so the structures are easily obtainable experimentally. 

% ORDERING OF COLLOIDS

To compose a bulk metamaterial from bottoms up, we would like to test self-assembly of metallic colloids in liquid crystals.
As some orientations of the colloid are energetically more favourable because of nematic elastic energy, the horseshoe colloids might, as the theoretical studies of J. Aplinc \cite{jure-HS} show, self-assemble in 2D and 3D photonic crystals. 
The analysis of the interplay between the influences of the geometry of such a photonic crystal and the optical parameters of the materials used provides a rich case of study. 
Comparison of metallic colloidal inclusions with dielectric ones also gives experimental diversity. 
Since the optical parameters are highly dependent on the frequency of light, diverse response is achieved across different wavelengths. 
Experimental realisation of these ideas would be explored in the team of prof. Muševič, group F5 on JSI. 

% \subsection{Waveguiding}
% 0.25 pages

% 
% 
% -- opis LC *splošen opis \cite{degennes}
%   -- dvolomnost *splošno \cite{hecht-optics}
%   -- elastičnost za self-assembly: zakaj urejenost *spomin v poroznih mat. \cite{tanaka-lc-memory-porous} 
%                                                    *urejanje koloidov eksperimentalno \cite{musevic-2013-assembly,smalyukh-2009-assembly} 
%                                                    *urejanje koloidov teoretično \cite{zumer-2012-colloidal-assembly}
%   -- razvoj waveguidinga *LC kot dodatek obstojecim opticnim vlaknom \cite{kitzerow-2014-lc-fibre,zografopoulos-2012-lc-fibre}, 
%                          *vlakna izkljucno iz LC \cite{cancula-2016-waveguiding}? 
% 
% -- opis MTM
%   -- kaj so to *review \cite{wegener-2011-nature-review}
%   -- kako maš tudi akustične MTM, pa tudi jih maš za različne valovne dolžine svetlobe (ni neg. n za vse lambda)
%   -- optične lastnosti: 
%     -- neg. lom \cite{schultz-2000-first-mtm,zhang-2008-fishnet}
%     -- t.i. plašči nevidnosti \cite{zhang-2015-skin-cloak}
%     -- leče z resolucijo pod valovno limito *metalenses at visible wavelengths: \cite{capasso-2016-metalens}
%     -- v zadnjem času bolj oblikovanje valovnih front (polarizacije, faze) \cite{capasso-2014-flat-optics-metasurface}
%     -- nadzor transmitivnosti *absorberji \cite{padilla-2012-mtm-absorbers}
%     -- metasurface *review Nature \cite{meinzer-2014-metasurface}
%     -- all-dielectric MTM *review Nature \cite{jacob-2016-all-dielectric}
%     -- hyperbolic *review Nature \cite{kivshar-2013-hyperbolic}
%     -- za urejanje direktorskega polja *s svetlobo čez metasurface in zakleneš direktorsko polje LC \cite{ozaki-2016-patterned-lc}
% 
% -- MTM with LC
%   -- neg. refraction classically *neg. lom v klasičnih LC \cite{lavrentovich-2006-lc-neg,assanto-2007-nematicons-lc-neg}
%   -- self-assembly from direct laser writing *npr. \cite{tartan-2017-dlw} -- bottoms up approach
%   -- metallic spheres in LC: disordered hyperbolic medium *dielektrične krogle \cite{xuan-2013-nanoparticle-lc,khoo-2014-nanoparticle-lc}
%   -- MTM building blocks shaped as LC to achieve order *ordered nanorods \cite{lavrentovich-2008-gold-nanorods,smalyukh-2010-self-alignment,lavrentovich-2009-nanorods} 
%                                                        *oblong plastic building blocks with embedded resonators \cite{shadrivov-2016-meta-liquid-crystal}
%                                                        *nanoparticle with LC molecules bound to it \cite{goodby-2011-lc-gold-mtm}
%   -- tuning: controlling of the resonant frequency *switch \cite{buchnev-2015-lc-mtm-switch}
%                                                    *tuning \cite{zhang-2007-lc-mtm-tuning}


% ------------ MY THESIS ------------


%
%
% ------------------- Methods -------------------------
% * zakaj se tega lotevamo v glavnem numerično: * zahtevni robni pogoji (komplicirana geometrija)
%                                               * materiali z netrivialnimi optičnimi lastnostmi (absorpcija, anizotropija)
% %
% % \subsection{FDTD}
% * finite difference: approximate the derivatives with finite differences (a discretization method)
% * time-domain: pokrije široko frekvenčno območje (prehodne pojave) z eno simulacijo (realna propagacija v času)
% * Yee in osnovna ideja FDTD (stacked grid, leapfrog time-stepping)
% * prednosti metode: dobiš točno sliko v času z vsemi prehodnimi pojavi zraven
% * slabosti: zna biti požrešna s spominom --) ker rabiš določeno število pik na valovno dolžino, so velikosti vzorcev, ki jih lahko simuliraš, omejene. 
% * prilagoditev celice za anizotropne materiale (tekoči kristali)
% * moja prilagoditev za frekvenčno disperzijo: ADE --) simuliranje permitivnosti realnih kovin z plazemskim modelom, Drudejevim modelom in Drude-Lorentzovim modelom
% * moja prilagoditev za frekvenčno disperzijo -- anizotropno --) simuliranje neg. n anizotropnih materialov
% 
% % \subsection{Modelling of liquid crystals -- Landau-de Gennes methods}
% * free energy minimisation relaxation algorithms 
% * Lattice Boltzmann methods? Najbrž bolj ne ... 
% 0.5 page
% 
%
%



We would like to cooperate with \ldots (našteješ vse, ki se jih spomniš (?))


% ----- Light propagation through photonic strutures ---
%
% 0.25 pages
% * Refraction on the boundary between vacuum and anisotropic homogeneous metamaterial
% * spatially modulated meta-liquid-crystal structures: xy -- primerjava optičnih lastnosti z že obstoječimi holesteričnimi tekočimi kristali
% * spatially modulated meta-liquid-crystal structures: xz -- kot zanimivost
% * spatially modulated other negative-n structures

% \subsection{Spatially modulated meta-liquid-crystal structures}
% liquid crystals as photonic crystals (xy-cholesterics, xz structures)
% 0.25 pages
% Photonic crystals, periodic systems
% spatially modulated structures (wacky propagation)
%
% % \subsection{Platelets}
% % 0.5 page
% As some orientations of the colloid are more favourable in terms of free energy, the horseshoe colloids might, as the theoretical studies of J. Aplinc \cite{jure-HS} show, self-assemble in 2D and 3D crystals. 
% The analysis of the interplay between the influences of the geometry of such a photonic crystal and the optical parameters of the materials used provides a rich case of study. 
% Since the optical parameters are highly dependent on the frequency of light, diverse response is achieved across different wavelengths. 
% Experimental realisation of these ideas would be explored in the team of prof. Muševič, group F5 on JSI. 
% 
% % \subsection{Waveguiding}
% % 0.25 pages
% 
% In cooperation with prof. Etienne Brasselet from University of Bordeaux, different proposals for waveguides, filled with liquid crystals, will be simulated and analysed theoretically. 
% Through different boundary conditions and other manipulation, different director profiles can be introduced into the liquid crystal, which can in turn affect the propagation of light. 
% With some director profiles we can achieve focusing; special emphasis will be put on the director profiles which exhibit focusing for positive anisotropy ($n_{o} < n_e$).
% % work with Brasselet
% % 
%
%
%
% 
% 

%
%
% ------------ Conclusion and outlook -----------------
%
%
% 0.5 page
% Broader context
% \newline
% TOTAL: 4--5 pages\\
% DONE: 0\\
% TO-DO: 4--5 pages\\

\bibliographystyle{ieeetr}
\bibliography{dispozicija}

\end{document}
