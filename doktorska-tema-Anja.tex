\documentclass[a4paper,11pt]{article}
\usepackage[utf8]{inputenc}
\usepackage[margin=1.3in]{geometry}
\usepackage{fullpage}
\usepackage{setspace}

%opening
\title{\Large{Doctoral thesis disposition} \\ \LARGE{Soft matter metamaterials}}
\author{Author: Anja Bregar \and Advisor: Miha Ravnik}

\begin{document}

\onehalfspacing

\maketitle

% \begin{abstract}
% In my PhD thesis, I will highlight different phenomena connected with light propagation in soft matter-like meta-materials. 
% \end{abstract}
% 
% 
% 
% ------------------- State of the art -----------------
%


% LIQUID CRYSTALS

Liquid crystals are soft materials with a degree of orienataional order in comparison with positional disorder. 
This is achieved by the molecules of liquid crystal being anisotropic, i.e. elongated in one direction, and also reinforced by intermolecular forces.  
In a range of temperatures, they exhibit an additional phase between solid in liquid, in which they retain a degree of orientational order while losing the positional order. 
They have proven to be an important material in soft matter physics with many possibilities for applications in different industries as well, ranging from liquid crystal displays to nematic ordering systems. 

As stated already, their main characteristic is a degree of orientational order combined with positional disorder (in nematics in 3D, at least partial -- 2D -- in smectic liquid crystal). 
Their orientational order leads to several other characteristics. 
First is their elasticity: their free energy increases, if -- because of external fields or boundary conditions -- their orientational order is disturbed. 
With the elasticity, they can take into account the orientational order of colloidal inclusions. 
On the other hand, their orientational order leads into optical anisotropy as well: mostly, liquid crystals possess an extraordinary index of refraction (along the long axis of the molecules) larger than the ordinary index. 
Their optical birefringence opens the possibility for exploitation in optical systems such as waveguides etc. 

They are also easily manipulated with external fields, especially electric field, which tends to turn the long axis of the liquid crystal molecules to be polarized along the field. 
With the electric field, the elastic forces and the optical properties can be controlled. 

The colloidal suspension in liquid crystals have been shown to be ordered because of the elasticity of liquid crystals \cite{1, 2, 3,}: 
Musevic and Smalyukh groups have experimentally shown the elastic forces between the defect structures to be strong enough to achieve orientational ordering of the (spatially anisotropic, i.e. square or triangular) colloids, and also to lower the free energy in such colloidal clusters by arranging. 
Theoretical work has been done by our group as well \cite{cc}. 

Optical applications include waveguides. 
Liquid crystals can be used as an addition to the existing photonic-crystal-like waveguides \cite{dd}, where it has been shown they increase the performance of such structure by XX\%. 
If the photonic fibre is filled exclusively with liquid crystal, it can be used as a waveguide as well, since the refractive-index gradient controls the flow-of-light. 



% METAMATERIALS


Metamaterials are another fascinating segment of materials science. 
They are characterised by their extraordinary optical properties \cite{gja}.
(We would have to add that the concept of metamaterials applies to any kind of waves; besides optical waves, also elastic or acoustic waves come into play). 

The defining feature of metamaterials is their man-made composition: they are built in periodic unit cells with their periodicity being generally much smaller than the wavelength of light. 
Since the building blocks are smaller than the wavelength, the light propagates through the material as if it was homogeneous (anisotropic, but homogeneous). 
However, since the geometry and the material composition of the building blocks can be adjusted, the optical response of such a composite can be, to an extent, engineered. 

The field of research has opened around year 2000 \cite{koak} and has so far expanded to different areas of investigation. 
The central interest at the beginning was to obtain materials which could produce negative refractive index for light of a given wavelength. 
It has been done for many different wavelengths \cite{pwqo}. 
Another dream was to fabricate an invisibility cloak or a lens with subwavelength resolution. 
Some of the work in that direction has been successfully done \cite{lal, ei}. 
Because of the high losses inside metals at optical frequencies, the incentive has moved from producing bulk metamaterials towards other options, namely dielectric metamaterials, which have lower losses \cite{}, and metasurfaces, which are 2D metamaterials. 
An use of the metamaterials is as selective absorbers for selected incoming angles, polarisations and wavelengths of light. 
Lately, the research focus has shifted towards wavefront shaping -- the shaping of the phase and the polarisation of the transmitted wave -- with metasurfaces \cite{}.  

Another direction in which the field has departed are hyperbolic metamaterials: materials with anisotropic permittivity tensor, which may also have some of its eigenvalues negative. 
With it, negative refraction can also be achieved in the sense of the Poynting vector being refracted negatively.


% LC WITH MTM 

Metamaterials are intrinsically connected with the field of liquid crystals, as both are optically anisotropic. 
In addition, the optical anisotropy could be in liquid crystal metamaterials enforced even further and more pronounced. 
Liquid crystals (and other birefringent materials) can by itself, without metamaterial inclusions, for a narrow region of incoming angles around perpendicular to the surface boundary, exhibit negative refraction, since the walk-off angle is negative for birefringent media with positive anisotropy. 

%It is not a true negative-index metamaterial. 
%Anisotropic metamaterials with hyperbolic dispersion relation can exhibit all possible angles of the Poynting vector. (to boš ti delala, ne o tem tu pisat.)

Another challenge in the field of metamaterials is the production of bulk metamaterials. 
In the past, the engineering of metamaterials has mostly taken the up-to-the-bottom approach, meaning that every building block has been specifically placed in its place, which is experimentally very difficult and renders many applications impractical. 
It would be very much desired to take a bottoms-up approach of self-arranging colloidal particles as in \cite{tartan}. 
There, the colloidal particles, obtained from direct laser writing, were self-assembled in liquid crystal. 
Some suggestions to obtain hyperbolic metamaterials have also shown the way of metallic (plasmonic) spheres, immersed in liquid crystal, which partially order due to the liquid crystal elasticity. 
Numerical analysis (homogenisation techniques) show that the eigenvalues of such media would be differently signed \cite{khoo}. 
Some experiments have been performed \cite{ponsinet-virginie?}. 

Metamaterial building blocks can also mimic liquid crystals in order to achieve their orientational order. 
Many experiments have been performed with golden nanorods ordering because of their prolong shape in connection with the elasticity of liquid crystals \cite{nanoparticles}. 
Some experiments have been very explicit in this mimicking \cite{shadrivov}. 

As it comes very naturally with liquid crystals, many applications also include the tuning of the director field with electric field. 
With the help of voltage, the resonant frequency, i.e. the operational frequency at which negative refractive index is observed, was fine-tuned, and the liquid crystals were used to switch between the on- and off-states of a metasurface. 




-- opis LC *splošen opis \cite{degennes}
  -- dvolomnost *splošno \cite{hecht-optics}
  -- elastičnost za self-assembly: zakaj urejenost *spomin v poroznih mat. \cite{tanaka-lc-memory-porous} 
                                                   *urejanje koloidov eksperimentalno \cite{musevic-2013-assembly,smalyukh-2009-assembly} 
                                                   *urejanje koloidov teoretično \cite{zumer-2012-colloidal-assembly}
  -- razvoj waveguidinga *LC kot dodatek obstojecim opticnim vlaknom \cite{kitzerow-2014-lc-fibre,zografopoulos-2012-lc-fibre}, 
                         *vlakna izkljucno iz LC \cite{cancula-2016-waveguiding}? 

-- opis MTM
  -- kaj so to *review \cite{wegener-2011-nature-review}
  -- kako maš tudi akustične MTM, pa tudi jih maš za različne valovne dolžine svetlobe (ni neg. n za vse lambda)
  -- optične lastnosti: 
    -- neg. lom \cite{schultz-2000-first-mtm,zhang-2008-fishnet}
    -- t.i. plašči nevidnosti \cite{zhang-2015-skin-cloak}
    -- leče z resolucijo pod valovno limito *metalenses at visible wavelengths: \cite{capasso-2016-metalens}
    -- v zadnjem času bolj oblikovanje valovnih front (polarizacije, faze) \cite{capasso-2014-flat-optics-metasurface}
    -- nadzor transmitivnosti *absorberji \cite{padilla-2012-mtm-absorbers}
    -- metasurface *review Nature \cite{meinzer-2014-metasurface}
    -- all-dielectric MTM *review Nature \cite{jacob-2016-all-dielectric}
    -- hyperbolic *review Nature \cite{kivshar-2013-hyperbolic}
    -- za urejanje direktorskega polja *s svetlobo čez metasurface in zakleneš direktorsko polje LC \cite{ozaki-2016-patterned-lc}

-- MTM with LC
  -- neg. refraction classically *neg. lom v klasičnih LC \cite{lavrentovich-2006-lc-neg,assanto-2007-nematicons-lc-neg}
  -- self-assembly from direct laser writing *npr. \cite{tartan-2017-dlw} -- bottoms up approach
  -- metallic spheres in LC: disordered hyperbolic medium *dielektrične krogle \cite{xuan-2013-nanoparticle-lc,khoo-2014-nanoparticle-lc}
  -- MTM building blocks shaped as LC to achieve order *ordered nanorods \cite{lavrentovich-2008-gold-nanorods,smalyukh-2010-self-alignment,lavrentovich-2009-nanorods} 
                                                       *oblong plastic building blocks with embedded resonators \cite{shadrivov-2016-meta-liquid-crystal}
                                                       *nanoparticle with LC molecules bound to it \cite{goodby-2011-lc-gold-mtm}
  -- tuning: controlling of the resonant frequency *switch \cite{buchnev-2015-lc-mtm-switch}
                                                   *tuning \cite{zhang-2007-lc-mtm-tuning}




% ------------ MY THESIS ------------



% \subsection{Platelets}
% 0.5 page
A way to engineer soft (meta-)materials is to combine liquid crystal with metallic colloids. 
The elasticity of liquid crystals affects the distribution and arrangement of colloids. 
As some orientations of the colloid are more favourable in terms of free energy, the horseshoe colloids might, as the theoretical studies of J. Aplinc \cite{jure-HS} show, self-assemble in 2D and 3D crystals. 
The analysis of the interplay between the influences of the geometry of such a photonic crystal and the optical parameters of the materials used provides a rich case of study. 
Since the optical parameters are highly dependent on the frequency of light, diverse response is achieved across different wavelengths. 
Experimental realisation of these ideas would be explored in the team of prof. Muševič, group F5 on JSI. 

% \subsection{Waveguiding}
% 0.25 pages
A crucial reason why to control and manipulate light is to guide it. 
Many different applications, from photonic circuits to waveguides, all designed to convey information, rely on efficient control of light. 
An compelling option for waveguiding is to enhance their properties with liquid crystals. 
In cooperation with prof. Etienne Brasselet from University of Bordeaux, different proposals for waveguides, filled with liquid crystals, will be simulated and analysed theoretically. 
Through different boundary conditions and other manipulation, different director profiles can be introduced into the liquid crystal, which can in turn affect the propagation of light. 
With some director profiles we can achieve focusing; special emphasis will be put on the director profiles which exhibit focusing for positive anisotropy ($n_{o} < n_e$).
% work with Brasselet
% 
%
%
%
% ----- Light propagation through photonic strutures ---
%
% 0.25 pages
% Refraction on the boundary between vacuum and anisotropic homogeneous metamaterial
% \subsection{Spatially modulated meta-liquid-crystal structures}
% liquid crystals as photonic crystals (xy-cholesterics, xz structures)
% 0.25 pages
% Photonic crystals, periodic systems
% spatially modulated structures (wacky propagation)
% 
% 
%
%
% ------------------- Methods -------------------------
%
% \subsection{FDTD}
* Yee in osnovna ideja FDTD
* prilagoditev celice za anizotropijo
* moja prilagoditev za frekvenčno disperzijo: ADE --> simuliranje realnih kovin
* moja prilagoditev za frekvenčno disperzijo -- anizotropno --> simuliranje neg. n anizotropnih materialov

0.25 page
% \subsection{Permittivity models}
% 0.25 page
% \subsection{Modelling of liquid crystals -- Landau-de Gennes methods}
* free energy minimisation: 
* Lattice Boltzmann methods? Najbrž bolj ne ... 
% 0.5 page
% 
%
%
%
%
% ------------ Conclusion and outlook -----------------
%
%
% 0.5 page
% Broader context
% \newline
% TOTAL: 4--5 pages\\
% DONE: 0\\
% TO-DO: 4--5 pages\\

\bibliographystyle{ieeetr}
\bibliography{dispozicija}

\end{document}
