\documentclass[a4paper,11pt]{article}
\usepackage[utf8]{inputenc}
\usepackage[margin=1.3in]{geometry}
\usepackage{fullpage}
\usepackage{setspace}

%opening
\title{\huge{Flow of light in metamaterials based on nematic fluids}}
\author{Author: Anja Bregar \and Advisor: Miha Ravnik}

\begin{document}

\setcounter{page}{11}

\onehalfspacing

\maketitle

% \begin{abstract}
% In my PhD thesis, I will highlight different phenomena connected with light propagation in soft matter-like meta-materials. 
% \end{abstract}
% 
% 
% 
% ------------------- State of the art -----------------
%


% METAMATERIALS


Optical metamaterials are a fascinating segment of materials science and are characterised by their extraordinary optical properties \cite{wegener-2011-nature-review}.
%(We would have to add that the concept of metamaterials applies to any kind of waves; besides optical waves, also elastic or acoustic waves come into play). 
The defining feature of metamaterials is their man-made composition: they are built in periodic unit cells with periodicity generally several times smaller than the wavelength of operating light waves, with the intent of light propagating through the material as if the material was homogeneous. 
However, since the specific geometric characteristics and the material composition of the periodic building blocks can be adjusted, the optical response of such a composite can be, to an extent, engineered. 
Since the early 2000s, the field of metamaterials' research has expanded to many different areas of investigation. 
The central interest at the beginning was to obtain materials which could produce negative refractive index for light in some chosen range of wavelengths, which has been experimentally observed for wavelengths from infrared to optical \cite{schultz-2000-first-mtm,zhang-2008-fishnet}. 
With negative-index materials, some attractive applications have been constructed, for instance lenses with subwavelength resolution or invisibility cloaks \cite{capasso-2016-metalens,zhang-2015-skin-cloak}. 
But because of the high losses inside metals at optical frequencies and fabrication difficulties, the incentive has lately moved from producing bulk metamaterials with metallic components towards other options, like metamaterials made from dielectrics \cite{jacob-2016-all-dielectric}, which have lower losses, and metasurfaces -- 2D metamaterials \cite{meinzer-2014-metasurface}. 
With metasurfaces, it is possible to precisely shape the wavefront -- design the phase and polarisation -- of the transmitted light \cite{capasso-2014-flat-optics-metasurface}, and use them as selective absorbers. 
While it is difficult to construct optically perfectly isotropic metamaterials, the anisotropy has also been exploited in the field of hyperbolic metamaterials \cite{kivshar-2013-hyperbolic}, which have an anisotropic permittivity tensor with some eigenvalues negative. 
With it, negative refraction of the Poynting vector can also be achieved without the need for negative permeability.



% LIQUID CRYSTALS

Liquid crystals are another major family of optical materials, today used widely in various optic and photonic applications, especially displayes. 
They are soft materials made from rod-like molecules, which in a range of temperatures and molecular concentrations posess a degree of orientational order in contrast to their positional disorder \cite{degennes}. 
Their orientational order is the origin of several other characteristics. 
One of them is their anisotropic elasticity: the free energy of liquid crystals increases, if -- because of external fields or boundary conditions -- their orientational order is disturbed. 
The structural and spatially dependent forces, which arise from that account, generate a plethora of field structures and ordering phenomena. 
%With the elasticity, they can take into account the orientational order of colloidal inclusions. 
Furthermore, their orientational order implies optical anisotropy \cite{hecht-optics}: mostly, liquid crystals possess an extraordinary index of refraction (along the long axis of the molecules) larger than the ordinary index. 
Their optical birefringence is an noteworthy tool in the control of flow-of-light in optical systems. 
Since liquid crystals are also relatively easily manipulated with external fields, especially electric field, which tends to turn the long axis of the liquid crystal molecules to be polarized along the field, the elastic and optical properties can be well controlled.
Today, liquid crystals are explored in various directions, such as the controlled assembly of colloidal crystals \cite{zumer-2006-assembly,musevic-2013-assembly}, research of complex topological states \cite{musevic-2011-knots,smalyukh-2014-knots,zumer-2014-knots} and microfluidics \cite{yeomans-2013-lc-microfluidics}.
In the broader context of photonics, diverse applications lie in the fields of sensing \cite{abbott-2013-sensing}, tunable lenses \cite{neyts-2017-lensing,lin-2011-lensing}, tunable waveguides \cite{kitzerow-2014-fibres}, and tunable lasers \cite{humar-2016-lasing}.

% LC WITH MTM 

The fields of liquid crystals and metamaterials are intertwined in a number of experimental and theoretical examples. 
It is worth mentioning negative refraction is not limited to metamaterials: liquid crystals and other birefringent materials can refract light negatively, when the incoming angle of light is almost perpendicular \cite{lavrentovich-2006-lc-neg}.
Cholesteric liquid crystal films have been used as metasurfaces because of their capability to efficiently control the phase of the transmitted light \cite{ozaki-2016-patterned-lc}.
Liquid crystals have been used to achieve orientational or even to an extent positional ordering of metamaterial constituent parts: numerous experiments have been performed with golden nanorods ordering because of their prolong shape in connection with the elasticity of liquid crystals \cite{lavrentovich-2008-gold-nanorods,smalyukh-2010-self-alignment,lavrentovich-2009-nanorods}. 
Numerical analysis has suggested that a material composed of metallic spheres, dispersed in liquid crystal, would have the eigenvalues of permittivity tensor signed oppositely \cite{xuan-2013-nanoparticle-lc,khoo-2014-nanoparticle-lc}. 
Experiments have been performed in that direction which show that coated gold nanospheres dispersed in nematic liquid crystals have a negative ordinary and positive extraordinary eigenvalues of the permittivity tensor at low frequencies \cite{goodby-2011-lc-gold-mtm}. 
Another usage of liquid crystals inside metamaterials was to adjust the metamaterial properties with electric field. 
With applying voltage, the operational frequency of the metamaterial can be fine-tuned \cite{zhang-2007-lc-mtm-tuning,shalaev-2007-tunable-lc,baets-2011-ring-resonators}, and the liquid crystals can be used to switch between the on- and off-states of a metasurface \cite{buchnev-2015-lc-mtm-switch}. 


%The influence of optical anisotropy of the host liquid crystal medium on metamaterial response is still an open question. 
%Likewise, colloidal assembly because of liquid crystal material anisotropy 
%Some of the open questions are influence of optical anisotropy on the metamaterial response and bottoms-up fabrications of metamaterials. 
%Even if some of the work has been already done, there is still many missing links between optical metamaterials and liquid crystals.
The primary motivation behind this thesis is to study the role of optical anisotropy on the propagation of the light fields through the metamaterial.
The anisotropy could be a feature of the metamaterial as a whole or a characteristic of some of its constituent parts. 
By studying the anisotropy, we are moving towards another very relevant open challenge: to create a controllable soft metamaterial, whose properties could be tuned with external fields and parameters. 
%Along the way we would like to propose a soft metamaterial, which would combine the best of the worlds of metamaterials and liquid crystals: acquire the possibility of controlling and the tunability of liquid crystals, and the optical control which can be achieved with metamaterials.
Complex nematic fluids, perhaps combined with colloidal inclusions, can achieve such anisotropic controllable structures. 
So the end goal of this thesis is to explore the possibilities for the implementation of tunable soft metamaterials. 
%Also, structures with such inclusions can be self-assembled, which would greatly alleviate the complexity of metamaterial fabrications.
%This thesis will try to give a response to these open and active topics of investigation. 
%This is the starting point of this thesis. 

% mnajka 1 odstavek problem identification..kaj so odprta vprasanja.
% 
% Motivacija je raziskati pomen opticne anizotropije na odziv metamtarail in po moznosti ustvariti controllable soft memtamaterial ki bo tunable .....
% 
% in sistemi s kolidnimi vkljucki, compelx nematic lahko to naredijo ....
% 
% In potem zakljucek da bo tvoj dr odgovor na to temo
% 
% Potem za tem odstavkom pa zacnes s stavkom.

%The goal of this  thesis is to explore the ... Specifically we wil address ...



% REFRACTION INTO HYPERBOLIC METAMATERIALS

Specifically we will address optically anisotropic systems with the eigenvalues of the permittivity tensor with different signs: liquid-crystal like metamaterials. 
We will be interested in a case of refraction from vacuum into a system with optical axis homogeneous across the material and tilted with respect to the metamaterial boundary. 
As an upgrade of these systems, examples with periodic modulation of the director profile will be examined. 
As a comparison with the already existing cholesteric positive-refractive-index systems, where the optical axis is twisting in the plane parallel to the material boundary, we will explore cholesteric hyperbolic metamaterial systems with some of the permittivity eigenvalues negative.
We will be especially interested in the Bragg-reflective regime for circularly polarised light. 
Additionally, we will perform simulations of the propagation of light at an angle with the optical axis, for which no clear theoretical framework has yet been established. 
In general, we will model spatial modulation of frequency-dispersive permittivity tensor and explore possibilities for successful manouvering of light propagation.  

% WAVEGUIDING 

% A crucial reason why to control and manipulate light is to guide it. 
% Many different applications, from photonic circuits to waveguides, all designed to convey information, rely on efficient control of light. 
% An compelling option for waveguiding is to enhance their properties with liquid crystals. 
% Liquid crystals can be used as an addition to the existing photonic-crystal-like waveguides \cite{dd}, where it has been shown they increase the performance of such structure by XX\%. 
% If the photonic fibre is filled exclusively with liquid crystal, it can be used as a waveguide as well, since the refractive-index gradient controls the flow-of-light. 

The control over the flow of light will be explored for materials with all-positive refractive index as well in the context of metamaterial applications, such as waveguides and lenses. 
An appealing option for waveguiding is to enhance the properties of waveguides with liquid crystals. 
In cooperation with prof. Etienne Brasselet from University of Bordeaux, different proposals for waveguides, filled with liquid crystals, will be simulated and analysed theoretically. 
Through different boundary conditions and other manipulation, different director profiles can be introduced into the liquid crystal waveguide. 
The modulation of the refractive index profile can in turn affect the propagation of light. 
With some director profiles we can achieve focusing of selected incoming polarizations.
Special emphasis will be put on the director profiles which exhibit focusing for positive anisotropy ($n_{o} < n_e$), so the structures are easily obtainable experimentally. 

% ORDERING OF COLLOIDS


% A way to engineer soft (meta-)materials is to combine liquid crystal with metallic colloids. 
% The elasticity of liquid crystals affects the distribution and arrangement of colloids. 
% The colloidal suspension in liquid crystals have been shown to be ordered because of the elasticity of liquid crystals \cite{1, 2, 3,}: 
% Musevic and Smalyukh groups have experimentally shown the elastic forces between the defect structures to be strong enough to achieve orientational ordering of the (spatially anisotropic, i.e. square or triangular) colloids, and also to lower the free energy in such colloidal clusters by arranging. 
% Theoretical work has been done by our group as well \cite{cc}. 

To auto-compose a bulk metamaterial from bottom scales up, we would like to assess self-assembly of metallic colloidal particles in liquid crystals.
As some orientations of the colloidal particles are energetically more favourable because of nematic elastic energy \cite{musevic-2013-assembly,smalyukh-2009-assembly}, the horseshoe-shaped colloids might self-assemble in 2D and 3D photonic crystals. 
The analysis of the interplay between the influences of the geometry of such a colloidal metamaterial and the optical parameters of the materials used for the colloids provides a rich case of study. 
Comparison of metallic colloidal inclusions with dielectric ones also gives experimental diversity. 
Since the optical parameters are highly dependent on the frequency of light, contrasting optical response is achieved across different wavelengths. 
Experimental realisation of these ideas would be explored in possible collaboration with the group of prof. Muševič at JSI. 

% KODA

Propagation of light through metamaterials is very complex due to various material characteristics and elaborate geometries of its constituent parts. 
In liquid crystals, the exact calculation of the flow of light can be, because of the complicated director field structures, equally challenging.
With that in mind, numerical simulations are a viable tool in determining the response of such soft matter structures. 
The calculations for my thesis will be performed with computer code based on the FDTD method, which was implemented in our group.
%FDTD (short for finite-difference time-domain) is a finite difference method, where the derivatives are approximated with finite differences, while the time-domain part means a wide range of frequencies can be covered with one simulation. (prehodni pojavi). 
In FDTD (short for finite-difference time-domain) \cite{taflove1}, Maxwell's equations are integrated quite explicitly, which allows for straightforward implementation of different optical phenomena.
The differential equations are integrated over Yee mesh, in which the components of electrical and magnetic fields are positioned in a cubic stacked grid and propagated in a leapfrog fashion in time. 
Our code is adapted for optically anisotropic materials, as well as for anisotropic and frequency dispersive materials with the adapted ADE (auxiliary differential equation \cite{taflove1}) method. 
As a model for frequency dispersive dielectric function of a material, the plasma, Drude, and Lorentz models can be used. 
%We have also implemented the possibility of anisotropic and frequency dispersive materials in one. 

While the main objective of this PhD is to explore the optical response of soft matter metamaterials, it is essential to take into account the relaxation of the liquid crystal because of the optical electric fields as well. 
This will be done with free energy minimisation algorithms, namely the relaxation algorithm \cite{ravnik-2009-lc-modelling}, also incorporated in our group. 
The code takes into account different time scales of the light propagation ($\approx \mathrm{ps}$) and relaxation ($\approx \mathrm{ms}$). 

To summarize, in the thesis we will consider conceptually different combinations of optical metamaterials and liquid crystals, linked together by the notion of optical anisotropy. 
On one hand we will deal with homogenised materials with exceptional properties, on the other hand we will aim to design a soft colloidal metamaterial from bottoms-up. 
%Optical anisotropy can arise from the design of metamaterials themselves, or by introduction of liquid crystal into the metamaterial.
With the work, we would like to propose a new perspective for the control and manipulation with light. 



% \subsection{Waveguiding}
% 0.25 pages

% 
% 
% -- opis LC *splošen opis \cite{degennes}
%   -- dvolomnost *splošno \cite{hecht-optics}
%   -- elastičnost za self-assembly: zakaj urejenost *spomin v poroznih mat. \cite{tanaka-lc-memory-porous} 
%                                                    *urejanje koloidov eksperimentalno \cite{musevic-2013-assembly,smalyukh-2009-assembly} 
%                                                    *urejanje koloidov teoretično \cite{zumer-2012-colloidal-assembly}
%   -- razvoj waveguidinga *LC kot dodatek obstojecim opticnim vlaknom \cite{kitzerow-2014-lc-fibre,zografopoulos-2012-lc-fibre}, 
%                          *vlakna izkljucno iz LC \cite{cancula-2016-waveguiding}? 
% 
% -- opis MTM
%   -- kaj so to *review \cite{wegener-2011-nature-review}
%   -- kako maš tudi akustične MTM, pa tudi jih maš za različne valovne dolžine svetlobe (ni neg. n za vse lambda)
%   -- optične lastnosti: 
%     -- neg. lom \cite{schultz-2000-first-mtm,zhang-2008-fishnet}
%     -- t.i. plašči nevidnosti \cite{zhang-2015-skin-cloak}
%     -- leče z resolucijo pod valovno limito *metalenses at visible wavelengths: \cite{capasso-2016-metalens}
%     -- v zadnjem času bolj oblikovanje valovnih front (polarizacije, faze) \cite{capasso-2014-flat-optics-metasurface}
%     -- nadzor transmitivnosti *absorberji \cite{padilla-2012-mtm-absorbers}
%     -- metasurface *review Nature \cite{meinzer-2014-metasurface}
%     -- all-dielectric MTM *review Nature \cite{jacob-2016-all-dielectric}
%     -- hyperbolic *review Nature \cite{kivshar-2013-hyperbolic}
%     -- za urejanje direktorskega polja *s svetlobo čez metasurface in zakleneš direktorsko polje LC \cite{ozaki-2016-patterned-lc}
% 
% -- MTM with LC
%   -- neg. refraction classically *neg. lom v klasičnih LC \cite{lavrentovich-2006-lc-neg,assanto-2007-nematicons-lc-neg}
%   -- self-assembly from direct laser writing *npr. \cite{tartan-2017-dlw} -- bottoms up approach
%   -- metallic spheres in LC: disordered hyperbolic medium *dielektrične krogle \cite{xuan-2013-nanoparticle-lc,khoo-2014-nanoparticle-lc}
%   -- MTM building blocks shaped as LC to achieve order *ordered nanorods \cite{lavrentovich-2008-gold-nanorods,smalyukh-2010-self-alignment,lavrentovich-2009-nanorods} 
%                                                        *oblong plastic building blocks with embedded resonators \cite{shadrivov-2016-meta-liquid-crystal}
%                                                        *nanoparticle with LC molecules bound to it \cite{goodby-2011-lc-gold-mtm}
%   -- tuning: controlling of the resonant frequency *switch \cite{buchnev-2015-lc-mtm-switch}
%                                                    *tuning \cite{zhang-2007-lc-mtm-tuning}


% ------------ MY THESIS ------------


%
%
% ------------------- Methods -------------------------
% * zakaj se tega lotevamo v glavnem numerično: * zahtevni robni pogoji (komplicirana geometrija)
%                                               * materiali z netrivialnimi optičnimi lastnostmi (absorpcija, anizotropija)
% %
% % \subsection{FDTD}
% * finite difference: approximate the derivatives with finite differences (a discretization method)
% * time-domain: pokrije široko frekvenčno območje (prehodne pojave) z eno simulacijo (realna propagacija v času)
% * Yee in osnovna ideja FDTD (stacked grid, leapfrog time-stepping)
% * prednosti metode: dobiš točno sliko v času z vsemi prehodnimi pojavi zraven
% * slabosti: zna biti požrešna s spominom --) ker rabiš določeno število pik na valovno dolžino, so velikosti vzorcev, ki jih lahko simuliraš, omejene. 
% * prilagoditev celice za anizotropne materiale (tekoči kristali)
% * moja prilagoditev za frekvenčno disperzijo: ADE --) simuliranje permitivnosti realnih kovin z plazemskim modelom, Drudejevim modelom in Drude-Lorentzovim modelom
% * moja prilagoditev za frekvenčno disperzijo -- anizotropno --) simuliranje neg. n anizotropnih materialov
% 
% % \subsection{Modelling of liquid crystals -- Landau-de Gennes methods}
% * free energy minimisation relaxation algorithms 
% * Lattice Boltzmann methods? Najbrž bolj ne ... 
% 0.5 page
% 
%
%



% We would like to cooperate with \ldots (našteješ vse, ki se jih spomniš (?))


% ----- Light propagation through photonic strutures ---
%
% 0.25 pages
% * Refraction on the boundary between vacuum and anisotropic homogeneous metamaterial
% * spatially modulated meta-liquid-crystal structures: xy -- primerjava optičnih lastnosti z že obstoječimi holesteričnimi tekočimi kristali
% * spatially modulated meta-liquid-crystal structures: xz -- kot zanimivost
% * spatially modulated other negative-n structures

% \subsection{Spatially modulated meta-liquid-crystal structures}
% liquid crystals as photonic crystals (xy-cholesterics, xz structures)
% 0.25 pages
% Photonic crystals, periodic systems
% spatially modulated structures (wacky propagation)
%
% % \subsection{Platelets}
% % 0.5 page
% As some orientations of the colloid are more favourable in terms of free energy, the horseshoe colloids might, as the theoretical studies of J. Aplinc \cite{jure-HS} show, self-assemble in 2D and 3D crystals. 
% The analysis of the interplay between the influences of the geometry of such a photonic crystal and the optical parameters of the materials used provides a rich case of study. 
% Since the optical parameters are highly dependent on the frequency of light, diverse response is achieved across different wavelengths. 
% Experimental realisation of these ideas would be explored in the team of prof. Muševič, group F5 on JSI. 
% 
% % \subsection{Waveguiding}
% % 0.25 pages
% 
% In cooperation with prof. Etienne Brasselet from University of Bordeaux, different proposals for waveguides, filled with liquid crystals, will be simulated and analysed theoretically. 
% Through different boundary conditions and other manipulation, different director profiles can be introduced into the liquid crystal, which can in turn affect the propagation of light. 
% With some director profiles we can achieve focusing; special emphasis will be put on the director profiles which exhibit focusing for positive anisotropy ($n_{o} < n_e$).
% % work with Brasselet
% % 
%
%
%
% 
% 

%
%
% ------------ Conclusion and outlook -----------------
%
%
% 0.5 page
% Broader context
% \newline
% TOTAL: 4--5 pages\\
% DONE: 0\\
% TO-DO: 4--5 pages\\

\bibliographystyle{amsplain_copar}
\bibliography{dispozicija}

\end{document}
