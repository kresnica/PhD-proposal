\documentclass[a4paper,10pt]{article}
\usepackage[utf8]{inputenc}
\usepackage[margin=1.3in]{geometry}

%opening
\title{\Large{Doctoral thesis disposition} \\ \LARGE{Soft matter metamaterials}}
\author{Author: Anja Bregar \and Advisor: Miha Ravnik}

\begin{document}

\maketitle

% \begin{abstract}
% In my PhD thesis, I will highlight different phenomena connected with light propagation in soft matter-like meta-materials. 
% \end{abstract}
% 
% 
% 
% ------------------- State of the art -----------------
%
-- opis LC *splošen opis \cite{degennes}
  -- dvolomnost *splošno \cite{hecht-optics}
  -- elastičnost za self-assembly: zakaj urejenost *spomin v poroznih mat. \cite{tanaka-lc-memory-porous} 
                                                   *urejanje koloidov eksperimentalno \cite{musevic-2013-assembly,smalyukh-2009-assembly} 
                                                   *urejanje koloidov teoretično \cite{zumer-2012-colloidal-assembly}
  -- razvoj waveguidinga *LC kot dodatek obstojecim opticnim vlaknom \cite{kitzerow-2014-lc-fibre,zografopoulos-2012-lc-fibre}, 
                         *vlakna izkljucno iz LC \cite{cancula-2016-waveguiding}? 

-- opis MTM
  -- kaj so to *review \cite{wegener-2011-nature-review}
  -- kako maš tudi akustične MTM, pa tudi jih maš za različne valovne dolžine svetlobe (ni neg. n za vse lambda)
  -- optične lastnosti: 
    -- neg. lom \cite{schultz-2000-first-mtm,zhang-2008-fishnet}
    -- t.i. plašči nevidnosti \cite{zhang-2015-skin-cloak}
    -- leče z resolucijo pod valovno limito *metalenses at visible wavelengths: \cite{capasso-2016-metalens}
    -- v zadnjem času bolj oblikovanje valovnih front (polarizacije, faze) \cite{capasso-2014-flat-optics-metasurface}
    -- nadzor transmitivnosti *absorberji \cite{padilla-2012-mtm-absorbers}
    -- metasurface *review Nature \cite{meinzer-2014-metasurface}
    -- all-dielectric MTM *review Nature \cite{jacob-2016-all-dielectric}
    -- hyperbolic *review Nature \cite{kivshar-2013-hyperbolic}
    -- za urejanje direktorskega polja *s svetlobo čez metasurface in zakleneš direktorsko polje LC \cite{ozaki-2016-patterned-lc}

-- MTM with LC
  -- neg. refraction classically *neg. lom v klasičnih LC \cite{lavrentovich-2006-lc-neg,assanto-2007-nematicons-lc-neg}
  -- self-assembly from direct laser writing *npr. \cite{tartan-2017-dlw} -- bottoms up approach
  -- metallic spheres in LC: disordered hyperbolic medium *dielektrične krogle \cite{xuan-2013-nanoparticle-lc,khoo-2014-nanoparticle-lc}
  -- MTM building blocks shaped as LC to achieve order *ordered nanorods \cite{lavrentovich-2008-gold-nanorods,smalyukh-2010-self-alignment,lavrentovich-2009-nanorods} 
                                                       *oblong plastic building blocks with embedded resonators \cite{shadrivov-2016-meta-liquid-crystal}
                                                       *nanoparticle with LC molecules bound to it \cite{goodby-2011-lc-gold-mtm}
  -- tuning: controlling of the resonant frequency *switch \cite{buchnev-2015-lc-mtm-switch}
                                                   *tuning \cite{zhang-2007-lc-mtm-tuning}




% 1--2 pages
% The field of soft matter is great. 
% But an interesting question would be \ldots how the light interacts with structures made from or interwoven with liquid crystals. 
% Especially interesting is to take into regard the birefringent nature of liquid crystals and to combine it with metamaterial anisotropy. 
% 
%
%
%
% ------------ Development of soft materials ------------



% \subsection{Platelets}
% 0.5 page
A way to engineer soft (meta-)materials is to combine liquid crystal with metallic colloids. 
The elasticity of liquid crystals affects the distribution and arrangement of colloids. 
As some orientations of the colloid are more favourable in terms of free energy, the horseshoe colloids might, as the theoretical studies of J. Aplinc \cite{jure-HS} show, self-assemble in 2D and 3D crystals. 
The analysis of the interplay between the influences of the geometry of such a photonic crystal and the optical parameters of the materials used provides a rich case of study. 
Since the optical parameters are highly dependent on the frequency of light, diverse response is achieved across different wavelengths. 
Experimental realisation of these ideas would be explored in the team of prof. Muševič, group F5 on JSI. 

% \subsection{Waveguiding}
% 0.25 pages
A crucial reason why to control and manipulate light is to guide it. 
Many different applications, from photonic circuits to waveguides, all designed to convey information, rely on efficient control of light. 
An compelling option for waveguiding is to enhance their properties with liquid crystals. 
In cooperation with prof. Etienne Brasselet from University of Bordeaux, different proposals for waveguides, filled with liquid crystals, will be simulated and analysed theoretically. 
Through different boundary conditions and other manipulation, different director profiles can be introduced into the liquid crystal, which can in turn affect the propagation of light. 
With some director profiles we can achieve focusing; special emphasis will be put on the director profiles which exhibit focusing for positive anisotropy ($n_{o} < n_e$).
% work with Brasselet
% 
%
%
%
% ----- Light propagation through photonic strutures ---
%
% 0.25 pages
% Refraction on the boundary between vacuum and anisotropic homogeneous metamaterial
% \subsection{Spatially modulated meta-liquid-crystal structures}
% liquid crystals as photonic crystals (xy-cholesterics, xz structures)
% 0.25 pages
% Photonic crystals, periodic systems
% spatially modulated structures (wacky propagation)
% 
% 
%
%
% ------------------- Methods -------------------------
%
% \subsection{FDTD}
* Yee in osnovna ideja FDTD
* prilagoditev celice za anizotropijo
* moja prilagoditev za frekvenčno disperzijo: ADE --> simuliranje realnih kovin
* moja prilagoditev za frekvenčno disperzijo -- anizotropno --> simuliranje neg. n anizotropnih materialov

0.25 page
% \subsection{Permittivity models}
% 0.25 page
% \subsection{Modelling of liquid crystals -- Landau-de Gennes methods}
* free energy minimisation
* 
% 0.5 page
% 
%
%
%
%
% ------------ Conclusion and outlook -----------------
%
%
% 0.5 page
% Broader context
% \newline
% TOTAL: 4--5 pages\\
% DONE: 0\\
% TO-DO: 4--5 pages\\

\bibliographystyle{ieeetr}
\bibliography{dispozicija}

\end{document}
